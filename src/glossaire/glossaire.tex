\chapter*{Glossaire Technique}
\addcontentsline{toc}{chapter}{Glossaire Technique}
\begin{multicols}{2}
    \section*{A}

    \subsection*{Angular}
    \begin{itemize}
        \item Un framework JavaScript qui utilise TypeScript pour le développement d'applications Frontend.
    \end{itemize}

    \subsection*{API CRUD}
    \begin{itemize}
        \item Un ensemble de services permettant de créer, lire, mettre à jour et supprimer des données (CRUD signifie Create, Read, Update, Delete).
    \end{itemize}

    \subsection*{ArgoCD}
    \begin{itemize}
        \item Un outil de déploiement continu pour Kubernetes.
    \end{itemize}

    \section*{B}

    \subsection*{Backend}
    \begin{itemize}
        \item La partie serveur du développement, non visible pour un utilisateur.
    \end{itemize}

    \section*{D}

    \subsection*{DevOps}
    \begin{itemize}
        \item Une méthodologie combinant le développement logiciel (Dev) et les opérations informatiques (Ops) pour améliorer la collaboration et l'efficacité.
    \end{itemize}

    \subsection*{Docker}
    \begin{itemize}
        \item Un logiciel qui utilise la technologie de conteneurisation pour permettre aux applications de fonctionner de manière cohérente sur différents environnements.
    \end{itemize}

    \subsection*{DNS}
    \begin{itemize}
        \item Système de noms de domaine, un service qui convertit les noms de domaine lisibles par les humains en adresses IP lisibles par les machines.
    \end{itemize}

    \section*{F}

    \subsection*{FrontEnd}
    \begin{itemize}
        \item Les éléments visibles par l'utilisateur dans une application ou un site web.
    \end{itemize}

    \subsection*{Framework}
    \begin{itemize}
        \item Une boîte à outils pour les développeurs qui fournit des fonctionnalités prêtes à l'emploi et des règles à suivre pour gagner du temps.
    \end{itemize}

    \section*{G}

    \subsection*{Git}
    \begin{itemize}
        \item Un système de contrôle de version qui permet de suivre les modifications apportées au code source.
    \end{itemize}

    \subsection*{GitHub}
    \begin{itemize}
        \item Un site web qui permet de stocker du code en utilisant la technologie Git.
    \end{itemize}

    \subsection*{GitHub Actions}
    \begin{itemize}
        \item Un service d'automatisation et de déploiement continu intégré à GitHub.
    \end{itemize}

    \subsection*{Go (Golang)}
    \begin{itemize}
        \item Un langage de programmation créé par Google, compilé et fortement typé, avec une syntaxe simple d'utilisation.
    \end{itemize}

    \subsection*{Google Cloud Artifact Registry}
    \begin{itemize}
        \item Un service de Google Cloud Platform pour stocker des images de conteneurs.
    \end{itemize}

    \subsection*{Google Cloud Build}
    \begin{itemize}
        \item Un service de Google Cloud Platform pour automatiser la construction d'applications.
    \end{itemize}

    \subsection*{Google Cloud DNS}
    \begin{itemize}
        \item Un service de Google Cloud Platform pour gérer et résoudre les noms de domaine.
    \end{itemize}

    \subsection*{Google Cloud Platform}
    \begin{itemize}
        \item Une suite de services cloud proposés par Google.
    \end{itemize}

    \subsection*{Google Cloud Run}
    \begin{itemize}
        \item Un service de Google Cloud Platform pour exécuter des conteneurs Docker.
    \end{itemize}

    \subsection*{Google Cloud Secret Manager}
    \begin{itemize}
        \item Un service de gestion des secrets de Google Cloud Platform pour stocker et gérer de manière sécurisée des informations sensibles.
    \end{itemize}

    \section*{H}

    \subsection*{Helm}
    \begin{itemize}
        \item Un outil de gestion des applications Kubernetes qui facilite le déploiement et la gestion des applications sur Kubernetes.
    \end{itemize}

    \section*{J}

    \subsection*{JavaScript}
    \begin{itemize}
        \item Un langage de programmation principalement utilisé pour créer des scripts sur les pages web afin de les rendre interactives.
    \end{itemize}

    \section*{K}

    \subsection*{Kubernetes}
    \begin{itemize}
        \item Un outil qui automatise le déploiement, la gestion et la mise à l'échelle des conteneurs d'applications.
    \end{itemize}

    \section*{M}

    \subsection*{MongoDB}
    \begin{itemize}
        \item Une base de données NoSQL utilisée pour stocker des informations.
    \end{itemize}

    \section*{N}

    \subsection*{Node.js}
    \begin{itemize}
        \item Un environnement qui permet d'exécuter du code JavaScript côté serveur.
    \end{itemize}

    \section*{S}

    \subsection*{Serveur Cloud}
    \begin{itemize}
        \item Un ordinateur virtuel qui exécute des applications sur Internet.
    \end{itemize}

    \subsection*{SSH (Secure Shell)}
    \begin{itemize}
        \item Un protocole de communication sécurisé utilisant un chiffrement asymétrique pour se connecter de manière sécurisée à un serveur.
    \end{itemize}

    \section*{T}

    \subsection*{Terraform}
    \begin{itemize}
        \item Un outil d'infrastructure en tant que code (IaC) qui permet de définir et de gérer des infrastructures cloud avec des fichiers de configuration.
    \end{itemize}

    \subsection*{TypeScript}
    \begin{itemize}
        \item Une surcouche de JavaScript qui ajoute des types statiques pour rendre le code plus sécurisé et maintenable.
    \end{itemize}

    \section*{U}

    \subsection*{URL}
    \begin{itemize}
        \item Uniform Resource Locator, une adresse web qui spécifie l'emplacement d'une ressource sur Internet.
    \end{itemize}

    \section*{Y}

    \subsection*{YAML}
    \begin{itemize}
        \item Un langage de sérialisation de données facile à lire et à écrire utilisé pour les fichiers de configuration.
    \end{itemize}
\end{multicols}

\clearpage