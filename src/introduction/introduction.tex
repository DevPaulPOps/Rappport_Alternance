\chapter*{Introduction}
\addcontentsline{toc}{chapter}{Introduction}

A la fin de ma première année en BUT informatique, j’ai eu la possibilité de poursuivre mon cursus en apprentissage.
La recherche d’une entreprise n’a pas été évidente mais au bout de plusieurs mois de recherche j’ai su saisir une très bonne opportunité qui correspondait à mes critères.
Cependant, je n’ai pas pris ma décision directement car l’entreprise se trouvant à Bordeaux il était clair qu’une grande partie de mon apprentissage se ferait en distanciel.
Après mûre réflexion et suite à de nombreuses discussions avec l’entreprise, j’ai su que le suivi allait être quotidien et que le projet avait beaucoup à m’apporter, j’ai donc accepté.

\bigskip

En effet, durant cette première année d’apprentissage, j’ai intégré l’Entreprise de Service Numérique (ESN) \textbf{WeVii} qui est située à Bordeaux.
Elle a pour modèle économique la prestation de service dans le numérique.
WeVii Bordeaux possède une filiale se nommant WeVii innovation.
Ainsi, j’ai intégré WeVii Bordeaux afin d’être formé pour sa filiale.

\bigskip

Pour ma part, j’interviens dans un des projets nommé, « TDC » (Tree Digital Cloud) qui est un projet de transformation digitale.
Mais pour cela j’ai dû être formé aux technologies du cloud tout en ayant une approche DevOps.

\bigskip

En somme, j'ai été formé aux technologies d’un projet afin d’y être intégré pleinement.
Globalement, les missions qui me sont confiées sont la réalisation de projets techniques, dans l’optique de me faire découvrir et progresser sur différentes technologies.

\bigskip

Depuis quelques années, j’ai pour projet de me diriger vers l’architecture cloud, pour devenir architecte cloud.
C’est un métier qui demande beaucoup de compétences car ses missions sont diverses.
Il doit savoir automatiser la gestion des infrastructures, développer des fonctionnalités pour ces dernières, mais aussi avoir des qualités de management afin de gérer des équipes.
J’aime l’idée de faire un métier pertinent, complet qui regroupe différents domaines, cela permet de ne pas se lasser.
J’ai accepté cette alternance car j’ai su que j’allais développer, ce qui me donne l’opportunité de savoir si j’apprécie réellement le développement et si le métier d’architecte cloud est vraiment ce qui me correspond.

\bigskip

Dans la suite de mon rapport, je vais vous présenter l’entreprise ainsi que les deux projets réalisés au cours de mon alternance.
En conclusion, nous ferons un point sur l’ensemble de cette année ainsi que sur les années futures.

\clearpage
