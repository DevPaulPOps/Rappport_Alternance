\chapter*{Conclusion}
\addcontentsline{toc}{chapter}{Conclusion}
\begin{center}
    Cette première expérience professionnelle m’a été bénéfique sur différents points.

    \bigskip

    Tout d’abord, d’un point de vue technique, car j’ai appris de nombreuses choses dans le domaine de l’informatique, mais aussi d’un point de vue personnel, car j’ai découvert la conscience professionnelle, la rigueur et l’importance de la régularité.
    Mes compétences et connaissances ont grandement augmenté durant l’année entre septembre 2023 et maintenant.

    \bigskip

    Le principe de l’alternance consiste à alterner entre étudier à l’iut Paris Rives de Seine et travailler chez WeVii.
    Selon moi ce principe s’est avéré très enrichissante des deux cotés.
    D’un côté l’iut apporte un enseignant transversal tel que le droit, l’anglais, le management… Mais aussi une approche très théorique de ce que l’on peut pratiquer en entreprise ceci permet de comprendre plus en profondeur la partie technique mais aussi de comprendre le fonctionnement de l’entreprise grâce aux matières transversales.
    De l’autre côté l’entreprise permet d’avancer davantage sur les points techniques et apporte une pratique régulière qui favorise l’apprentissage.

    \bigskip

    Pour ma part, l’alternance m’a permis d’évoluer en termes de compétences mais aussi en connaissances et ainsi de constater plus facilement mon évolution personnelle.

    \bigskip

    Le fait de constater sa propre évolution personnelle entre chaque projet permet de s’intéresser d’autant plus à ces derniers.
    Dans l’informatique les projets sont nombreux et peuvent être très variés.
    Ce que j’aime dans ce domaine, c’est la pratique constante qui permet d’évoluer chaque jour.
    Grâce à cela, nous pouvons constater une réelle évolution tout en continuant d’apprendre quotidiennement, peu importe le projet ou le domaine.

    \bigskip

    À la suite de cette année, je souhaite poursuivre en école d'ingénieur, toujours en tant qu'alternant chez WeVii, tout en me spécialisant dans le domaine du DevOps avec un focus sur le Cloud.
    Cependant, certains aspects me restent à découvrir, comme le travail dans une grande équipe composée de développeurs, de designers, de testeurs et de managers.
    J'aimerais appliquer d'autres méthodes de travail et en particulier les méthodes agiles.
    Je souhaite réaliser des projets avec des contraintes plus importantes que celles de cette année, car je pense que cela serait une expérience enrichissante et m'apporterait de nouvelles compétences.

    \bigskip

    Plus tard, dans ma vie professionnelle, je souhaite continuer dans l'architecture cloud et le DevOps.
    Cette année a renforcé mes choix.
    Je voudrais faire un peu de salariat, mais aussi être indépendant, pour avoir du temps pour enseigner dans des domaines techniques, sous forme de conférences ou de cours.
    Je pense que l'enseignement pourrait me plaire, mais je ne me vois pas y consacrer toute ma vie.
    Cependant, enseigner entre 10 et 15 heures par semaine pourrait être une expérience très enrichissante.

\end{center}
\clearpage